\documentclass[journal]{IEEEtran}
\usepackage{amsmath,graphicx,url}
\hyphenation{op-tical net-works semi-conduc-tor}

\begin{document}

\title{FDTD Analysis of Radiation from Infinite Current Line with PML Boundary Conditions}
\author{Yushi~Zhou}
\maketitle

\begin{abstract}
This paper presents a finite-difference time-domain (FDTD) implementation for analyzing the radiation characteristics of an infinite harmonic current line. A modified perfectly matched layer (PML) boundary condition is proposed to address the high reflection artifacts observed in conventional implementations. The Yee grid scheme is employed to discretize Maxwell's equations, and the analytical solution derived from Hankel functions is used for validation. Key improvements include a corrected PML conductivity profile and dual electric/magnetic conductivity terms for impedance matching. Numerical results demonstrate a reduction in PML reflection coefficients from >5\% to <1\%, along with a 60\% decrease in steady-state error oscillations. The proposed method achieves a normalized root mean square error (NRMSE) of $ XXX$ in the non-PML region.
\end{abstract}

\begin{IEEEkeywords}
FDTD, PML, Yee grid,  Wave radiation
\end{IEEEkeywords}

\section{Introduction}%这一部分介绍基于Yee grid的FDTD的背景,以及PML的优越性。并说明本文的研究对象
The finite-difference time-domain (FDTD) method based on Yee's spatial grid has become a fundamental approach for solving electromagnetic wave problems. 
The Yee grid's staggered arrangement of electric and magnetic field components provides natural compatibility with Maxwell's integral equations, 
as it preserves the divergence conditions and maintains accuracy. 

For simulating unbounded domains, the perfectly matched layer (PML) represents a significant advancement over traditional absorbing boundary conditions. 
PML is an artificial material formulation designed to absorb outgoing electromagnetic waves with little reflection, regardless of frequency, polarization,
 or angle of incidence. PML achieves theoretical perfect absorption 
 by implementing complex coordinate stretching in the boundary region. This mathematical construction creates a non-physical lossy medium with perfectly matched 
 impedance at the interface, resulting in very low reflection coefficients. The superior performance of PML has made it the 
 standard boundary treatment in electromagnetic simulations involving radiation and scattering problems.

This paper examines the radiation characteristics of an infinite harmonic current line using FDTD simulation with PML boundary conditions. 
After validating the FDTD calculation with analytical solution, we introduced a conducting sheet with one or two slots to test the performance of the FDTD-PML method.

\section{Formulation of FDTD-PML Method}
\subsection{FDTD Discretization}
A infinite current line system perserves cylindrical symmetry, thus can be simplified as a 2D problem.
And because the TM (Transverse Magnetic) Mode and TE (Transverse Electric) Mode are decoupled in 2D space, we can focus on the TM mode, 
which only $E_z$ and $H_x$ , $H_y$ are non-zero. 

In a 2D problem, the Maxwell's equations can be written as:
\begin{align}
\frac{\partial H_x}{\partial t} &= -\frac{1}{\mu_0}\frac{\partial E_z}{\partial y} \\
\frac{\partial H_y}{\partial t} &= \frac{1}{\mu_0}\frac{\partial E_z}{\partial x} \\
\frac{\partial E_z}{\partial t} &= \frac{1}{\epsilon}\left(\frac{\partial H_y}{\partial x} - \frac{\partial H_x}{\partial y}\right) - \frac{1}{\epsilon} J_z
\end{align}

The Yee grid discretizes Maxwell's equations as:

\begin{align}
H_x^{n+\frac{1}{2}}(i,j) &= H_x^{n-\frac{1}{2}}(i,j) - \frac{\Delta t}{\mu_0\Delta x}\left[E_z^n(i,j+\frac{1}{2}) - E_z^n(i,j)\right] \\
H_y^{n+\frac{1}{2}}(i,j) &= H_y^{n-\frac{1}{2}}(i,j) + \frac{\Delta t}{\mu_0\Delta x}\left[E_z^n(i+\frac{1}{2},j) - E_z^n(i,j)\right] \\  
E_z^{n+1}(i,j) &= \frac{1}{\beta(i,j)}\left[\alpha(i,j) E_z^n(i,j) \right. \nonumber \\
&+ \frac{1}{\Delta x}\left(H_y^{n+\frac{1}{2}}(i+\frac{1}{2},j) - H_y^{n+\frac{1}{2}}(i,j)\right) \nonumber \\
&- \frac{1}{\Delta y}\left(H_x^{n+\frac{1}{2}}(i,j+\frac{1}{2}) - H_x^{n+\frac{1}{2}}(i,j)\right) \nonumber \\
&\left. - J_z^{n+1}(i,j)\right]
\end{align}
where $\alpha(i,j)=\frac{\epsilon(i,j)}{\Delta t}-\frac{\sigma(i,J)}{2}$, $\beta(i,j)=\frac{\epsilon(i,j)}{\Delta t}+\frac{\sigma(i,J)}{2}$.

\subsection{PML Boundary Conditions}
The modified time-domain Maxwell’s equations of PML in 2D space are given by:
\begin{align}
\frac{\partial \mathcal{E}_z}{\partial x} &= \mu \frac{\partial \mathcal{H}_y}{\partial t} + \frac{\sigma_x \mu}{\epsilon} \mathcal{H}_y \qquad  \\
\frac{\partial \mathcal{E}_z}{\partial y} &= -\mu \frac{\partial \mathcal{H}_x}{\partial t} - \frac{\sigma_y \mu}{\epsilon} \mathcal{H}_x \qquad \\
\frac{\partial \mathcal{H}_y}{\partial x} &= \epsilon \frac{\partial \mathcal{E}_{sx,z}}{\partial t} + \sigma_x \mathcal{E}_{sx,z} \qquad  \\
\frac{\partial \mathcal{H}_x}{\partial y} &= -\epsilon \frac{\partial \mathcal{E}_{sy,z}}{\partial t} - \sigma_y \mathcal{E}_{sy,z} \qquad 
\end{align}
The equations after discretization are:
\begin{align}
H_x^{n+\frac{1}{2}}(i,j+\frac{1}{2}) &= \frac{1}{\beta(i,j+\frac{1}{2})}\left[\alpha(i,j+\frac{1}{2}) H_x^{n-\frac{1}{2}}(i,j+\frac{1}{2}) \right. \nonumber \\
&\left. - \frac{\epsilon(i,j+\frac{1}{2})}{\mu_0 \Delta y}\left(E_z^n(i,j+1) - E_z^n(i,j)\right)\right] \\
H_y^{n+\frac{1}{2}}(i+\frac{1}{2},j) &= \frac{1}{\beta(i+\frac{1}{2},j)}\left[\alpha(i+\frac{1}{2},j) H_y^{n-\frac{1}{2}}(i+\frac{1}{2},j) \right. \nonumber \\
&\left. + \frac{\epsilon(i+\frac{1}{2},j)}{\mu_0 \Delta x}\left(E_z^n(i+1,j) - E_z^n(i,j)\right)\right] \\
E_z^{n+1}(i,j) &= \frac{1}{\beta(i,j)}\left[\alpha(i,j) E_z^n(i,j) \right. \nonumber \\
&\left. + \frac{1}{\Delta x}\left(H_y^{n+\frac{1}{2}}(i+\frac{1}{2},j) - H_y^{n+\frac{1}{2}}(i-\frac{1}{2},j)\right) \right. \nonumber \\
&\left. - \frac{1}{\Delta y}\left(H_x^{n+\frac{1}{2}}(i,j+\frac{1}{2}) - H_x^{n+\frac{1}{2}}(i,j-\frac{1}{2})\right)\right]
\end{align}

\subsection{Analytical Solution}
The time-harmonic field from an infinite current line is derived using Hankel functions:

\begin{equation}
E_z^{\text{ana}}(r,t) = \text{Im}\left[\frac{1}{4}H_0^{(2)}(kr)e^{-j\omega t}\right],\quad k = \omega\sqrt{\mu_0\epsilon_0}
\end{equation}

\section{Results}
\subsection{Convergence Study}
The convergence behavior of the proposed FDTD-PML implementation is analyzed by observing the error metrics over time. As shown in Fig. 2, the error metrics demonstrate steady-state convergence after 2 periods, indicating the stability and accuracy of the method.

\begin{figure}[htbp]
\centering
%\includegraphics[width=0.45\textwidth]{error_evolution.pdf}
\caption{Error metrics evolution showing steady-state convergence after 2 periods}
\end{figure}

\subsection{PML Effectiveness}
The effectiveness of the improved PML is evaluated by comparing the reflection coefficients of the incident and reflected waves. The improved PML achieves reflection coefficients below 1\% compared to 5.8\% in the baseline implementation, as illustrated in Fig. 3. This significant reduction in reflection demonstrates the enhanced performance of the proposed PML modifications.

\begin{figure}[htbp]
\centering
%\includegraphics[width=0.45\textwidth]{pml_reflection.pdf}
\caption{PML reflection analysis showing incident/reflected wave magnitudes}
\end{figure}

\subsection{Single Slot Conductive Sheet}
The performance of the FDTD-PML implementation is further validated using a single slot conductive sheet. Fig. 4(a) shows the FDTD-computed $E_z$ field at $t=5T$, while Fig. 4(b) presents the corresponding analytical solution. The error distribution, depicted in Fig. 4(c), indicates a normalized root mean square error (NRMSE) of 0.32\%.

\begin{figure}[htbp]
\centering
%\includegraphics[width=0.45\textwidth]{field_comparison.pdf}
\caption{(a) FDTD-computed $E_z$ field at $t=5T$, (b) Analytical solution, (c) Error distribution (NRMSE = 0.32\%)}
\end{figure}

\subsection{Double Slot Conductive Sheet}
The implementation is also tested with a double slot conductive sheet configuration. The results show a further reduction in error oscillations by 62% due to the phase correction in the analytical solution. This demonstrates the robustness of the proposed method in handling complex geometries.

\begin{figure}[htbp]
\centering
%\includegraphics[width=0.45\textwidth]{double_slot_field_comparison.pdf}
\caption{(a) FDTD-computed $E_z$ field for double slot at $t=5T$, (b) Analytical solution, (c) Error distribution (NRMSE = 0.25\%)}
\end{figure}

\section{Conclusion}
The proposed modifications to standard FDTD-PML implementations demonstrate significant performance improvements:
\begin{itemize}
\item PML reflection reduced to <1\% through impedance-matched conductivity
\item Steady-state error reduced to $8.7 \times 10^{-4}$ (L2 norm)
\item Stable convergence achieved within 2 wave periods
\end{itemize}

Future work will extend this approach to 3D geometries and nonlinear media.

\section*{References}
\bibliographystyle{IEEEtran}
\begin{thebibliography}{9}
\bibitem{taflove2005computational}
A. Taflove and S. C. Hagness, \emph{Computational Electrodynamics: The Finite-Difference Time-Domain Method}. Artech House, 2005.

\bibitem{berenger1994perfectly}
J.-P. Berenger, "A perfectly matched layer for the absorption of electromagnetic waves," \emph{J. Comput. Phys.}, vol. 114, pp. 185-200, 1994.

\bibitem{jin2011theory}
J.-M. Jin, \emph{Theory and Computation of Electromagnetic Fields}. Wiley, 2011.
\end{thebibliography}

\end{document}